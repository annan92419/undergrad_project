% \chapter{INTRODUCTION}

\begin{sloppypar}

		\section{Background}
		Renting a hostel is undoubtedly one of the most important decisions a student makes during their entire stay on campus. The price of these hostels depends on a wide variety of factors, ranging from location to the number of beds in a room, access to the shuttle, and many more. As the population of students increases, traditional hostel price predictions based on hostel price comparisons lacking an accepted standard can no longer be employed to estimate hostel prices. Therefore, the availability of a hostel price prediction model helps fill an important information gap \citep{Calhoun2003} and boosts the hostel market efficiency. \ac{ml} is one of the cutting-edge techniques that can be utilized to identify, interpret, and analyze hugely complicated data structures and patterns \citep{Ngiam2019}.
		
		\subsection{Machine Learning Overview}
		Machine Learning \citep{Douglass2020} is the science of programming computers to learn from data. The three major types of \ac{ml} are supervised, unsupervised, and reinforcement learning. In supervised learning, the \ac{ml} is guided with desired inputs and outputs by a human operator; the algorithm learns from the data and makes predictions. A typical supervised learning task is regression. Unsupervised learning algorithms find insight in unlabeled training data and organize the data in some way to describe its structure; a typical task is clustering. Reinforcement learning \citep{Youplus2018} is a type of machine learning technique that enables an agent to learn in an interactive environment by trial and error using feedback from its actions and experiences; the commonly used algorithm is Q-learning.
		
		
		\section{Problem Statement}
		In developed countries, \ac{ml} has been implemented successfully to estimate real estate prices. However, in Ghana, the application of \ac{ml} is rare and few and far between on the topic of hostels. The lack of adequate data \citep{Owusu-Ansah2012} has made it difficult, if not impossible, to develop efficient or systematic hostel pricing policies through modeling the hostel market. As a result, hostel managers always overprice their hostel rooms. This study aims to derive valuable insight into KNUST's hostel market through analyzing a real historical dataset. It seeks useful models to illustrate how \ac{ml} algorithms can be utilized to predict hostel prices given a set of its features.
		
		
		\section{Significance}
		Our models, if accepted, could allow students or hostel managers to make better decisions. In addition, it could benefit the projection of future hostel prices and the policy-making process for the hostel market.
		
		
		\section{Limitations}
		\begin{itemize}%[leftmargin=*]
			\item This paper proves the competence of \ac{ml} algorithms in the hostel market. Although it aims to assist policymakers in projecting future hostel prices, it does not, however, explicitly forecast hostel prices.
			
			\item 70 out of 105 recommended hostels by \ac{knust}; were considered for this study.
			
			\item Although several students rent homes (homestels), this study only focused solely on the prices of private hostels around KNUST. 
		\end{itemize}	
		
		\section{Outline}
		The next parts of this paper are constructed as follows: In Chapter 2, existing literature on housing market prediction applying different \ac{ml} algorithms will be reviewed. Chapter 3 explores the dataset, explains how to transform it into cleaned data, and introduces the various supervised \ac{ml} algorithms implemented. Chapter 4 presents our empirical results and the conclusion deduced in Chapter 5.
			
\end{sloppypar}