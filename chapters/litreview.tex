% \chapter{LITERATURE REVIEW}

\begin{sloppypar}
	
	\section{Introduction}
	Previous studies on the real estate market using \ac{ml} approaches can be categorized into two groups: the trend forecasting of house price indexes and house price valuations \citep{Phan2019}. Since house prices are strongly correlated to other factors such as location, area, and population \citep{Kamal2021b}, it requires other information apart from the house price index to predict individual house prices \citep{Truong2020}.
	
	
	\section{Related Work}
	\citep{Gavu2019a}, one of the very few papers to model the residential rental housing market in Ghana, implemented the hedonic price model to estimate house prices in the Accra Metropolis, Adenta, Ga East, La Dade Kotopon, and La Nkwantanang Madina areas while exploring the existence of submarkets. \newline
	\citep{panhouse} implemented and compared the performance of Ridge, Lasso, Multiple Linear Regression, Neural Nets, and Random Forest. In their findings, neural networks proved to be the most accurate in estimating house value. \newline
	\citep{Selim2009} and \citep{Limsombunchai2004a} papers offer a comparative approach between hedonic model and artificial neural network, although, on different datasets, results show that artificial neural network performs significantly better. On the other hand, \citep{Nguyen2001} compared multiple linear regression and artificial neural networks. Empirical results show that the performance of artificial neural networks improved as the data size increased. \newline
	$ L_1 $ and $ L_2 $ regularization was implemented by \citep{Xin2018} on a housing dataset from 2006 to 2010, lasso regression ($ L_1 $) produced much better predictions; while \citep{Madhuri2019}, involved Multiple Linear, Elastic Net, Gradient Boosting, and Ada Boost regression together with $ L_1 $ and $ L_2 $ regularization for estimating house price value; with a score of approximately 0.92, Gradient Boosting proved superior. \newline
	\citep{Thamarai2020a} applied Multiple Linear Regression and Decision Tree Regression to a housing dataset. Comparatively, the performance of multiple linear regression produced better results. \newline 
	\citep{Vineeth2018} modeled a Kaggle housing dataset utilizing 19 regression algorithms to help consumers find a price for their soon-to-be house without consulting a real estate agent (broker). Cat boost was the best performing model based on RMSE, with a value of $2.604e+04 $.

\end{sloppypar}